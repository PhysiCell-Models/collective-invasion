\documentclass[11pt]{article}
\newcommand{\Version}{1.2.0} 

\usepackage[letterpaper,margin=0.5in,bottom=0.75in]{geometry}
\usepackage{amsmath}
\usepackage{amssymb}
\usepackage{dsfont}
\usepackage{bbm}
\usepackage{stmaryrd}
\usepackage{graphicx}
\usepackage{pbox}
\usepackage{hyperref}
\usepackage{changepage}
\usepackage[dvipsnames,usenames]{color}
%\usepackage{biblatex}

\usepackage{caption}
\captionsetup{justification=raggedright,singlelinecheck=false}

\usepackage{tikz}
\usetikzlibrary{arrows,automata,arrows.meta}

\usepackage{mdframed}

\mdfdefinestyle{mystyle}{
    backgroundcolor=lightgray!10, 
    innertopmargin=5mm, 
    innerbottommargin=5mm, 
    innerleftmargin=5mm, 
    innerrightmargin=5mm 
}


\usepackage{bold-extra}


 
\usepackage[square,comma,numbers,sort&compress]{natbib}
\usepackage[nottoc]{tocbibind}

\usepackage{amsmath}
\usepackage{amssymb}
\usepackage{amsbsy}
\usepackage{color}
% \usepackage{mathbb}

\newcommand{\reals}{\mathbb{R}}




\newcommand{\braces}[1]{\left\{#1\right\}}
\newcommand{\N}{\mathcal{N}}

\newcommand{\Heaviside}[1]{\textrm{H}\left(#1\right)}

\newcommand{\beq}{\begin{equation}}
\newcommand{\eeq}{\end{equation}}

\newcommand{\beqa}{\begin{eqnarray}}
\newcommand{\eeqa}{\end{eqnarray}}
\newcommand{\mean}[1]{\langle #1 \rangle}
\newcommand{\abs}[1]{\left|#1\right|}
\newcommand{\plus}[1]{\left(#1\right)^+}
\newcommand{\minus}[1]{\left(#1\right)^-}

\renewcommand{\vec}[1]{\mathbf{#1}}
\newcommand{\grvec}[1]{\boldsymbol{#1}}

%\newcommand{\braces}[1]{\left\{#1\right\}}
%\newcommand{\abs}[1]{\left|#1\right|}
\renewcommand{\vec}[1]{\mathbf{#1}}
%\newcommand{\mean}[1]{\langle #1 \rangle}
\newcommand{\norm}[1]{\left|\left|{#1}\right|\right|}

%\newcommand{\Heaviside}[1]{\textrm{H}\left(#1\right)}
\renewcommand{\vec}[1]{\mathbf{#1}}
%\newcommand{\grvec}[1]{\boldsymbol{#1}}

%\newcommand{\beq}{\begin{equation}}
%\newcommand{\eeq}{\end{equation}}
%\newcommand{\beqa}{\begin{eqnarray}}
%\newcommand{\eeqa}{\end{eqnarray}}
\newcommand{\beqaN}{\begin{eqnarray*}}
\newcommand{\eeqaN}{\end{eqnarray*}}

\newcommand{\micron}{\mu\textrm{m}}
\newcommand{\specialcell}[2][c]{%
  \begin{tabular}[#1]{@{}c@{}}#2\end{tabular}}

\newcommand{\boldrho}{\rho\hspace{-5.1pt}\rho}
% \newcommand{\grvec}[1]{#1\hspace{-6pt}#1\hspace{-6pt}#1\hspace{1pt}}

\newcommand{\one}{\mathds{1}}

\newcommand{\hp}{\circ}%{\odot}%{\circ}
\newcommand{\hd}{\hspace{-1.6mm}\fatslash}%{\oslash}'

\setlength{\parskip}{1em}
\setlength{\parindent}{0in}

\newcommand{\code}[1]{\verb|#1|}
\renewcommand{\v}{\verb}
\renewcommand{\t}[1]{\left[\mathrm{#1}\right]}
\renewcommand{\tt}[1]{{\small \left[\texttt{#1}\right] }}

\newcommand{\smallcode}[1]{\textbf{\texttt{#1}}} 
%\newcommand{\smallcode}[1]{\textbf{\v|#1|}} 

\newcommand{\red}[1]{\textcolor{red}{#1}}
\newcommand{\blue}[1]{\textcolor{blue}{#1}}
\newcommand{\magenta}[1]{\textcolor{magenta}{#1}}
\newcommand{\FIX}{}%{\textbf{\red{[FIX ME]}}}
\newcommand{\DONE}{}%{\textbf{\blue{[DONE]}}}
\newcommand{\FINISH}{}%{\magenta{[FINISH ME]}}

%\newcommand{\beqa}{\begin{eqnarray}}
%\newcommand{\eeqa}{\end{eqnarray}}
%\newcommand{\beq}{\begin{equation}}
%\newcommand{\eeq}{\end{equation}}

%\newcommand{\abs}[1]{\left|#1\right|}
%\newcommand{\norm}[1]{\left|\left|#1\right|\right|}
\newcommand{\ip}[2]{\left\langle#1,#2\right\rangle}

%\newcommand{\Heaviside}[1]{\textrm{H}\left(#1\right)}

\newcommand{\oxy}{\textrm{pO}_2}
\newcommand{\oxyS}[1]{\textrm{pO2}_{2,\textrm{#1}}}

\renewcommand{\vec}[1]{\mathbf{#1}}
\renewcommand{\max}[2]{\textrm{max}\left(#1,#2\right)}
\renewcommand{\min}[2]{\textrm{min}\left(#1,#2\right)}

\newcommand{\TOClink}{\begin{center}\hrule\vskip-10pt\phantom{.}\hfill[Return to \hyperlink{TOC}{Table of Contents}.]\hfill\phantom{.}\vskip3pt\hrule\end{center}}
\begin{document}


\author{TBD ... }
\title{Modeling Extracellular-Cell Interactions using PhysiCell
}
\date{\today}

\maketitle

\section{Introduction}\label{sec:Intro}

ECM has three components 

- density ($\rho$) which represents amount of fiber present (runs from 0 to 1)

-anisotropy (a) which represents alignment of the fibers in a voxel with each other (runs from 0 to 1)

-fiber alignment ($\vec f$) which represents the average fiber alignment in a voxel - a normalized three component vector

General notes on cell motility in PhysiCell:

Let b be the migration bias with b such that 0 <= b <= 1 is bias parameter. 

Let $\vec B$ be the normalized migration bias vector and $\vec D$ be the migration direction. 

Then migration direction $\vec D$ is

\beq
\vec D = (1-b) \vec R + b \vec B
\eeq

where $\vec R$ is a normalized vector with 3 random components.

Letting $\vec V$ be the cell velocity vector defined as:

\beq
\vec V = \frac{s_{cell}\vec D}{norm(\vec D)}
\eeq

where $s_{cell}$ is the cell speed.

\section{Initialization}\label{sec:init}

Density initialized to 0.5. This value produces no alteration to cell speed. 

Anisotropy initialized to 0.

Fiber alignment initialized to a random orientation.

\section{Math Models}\label{sec:math_models}

\subsection{Cells Modifying ECM}\label{sec:cells_to_ecm}

Density: 

\beq
\frac{d\rho}{dt} = r_{density}
\eeq

where r$_{density}$ is a characteristic rate of growth of ECM density caused by a passing cell. Every cell in theory updates the density in its voxel in a growth like fashion in this version but currently r$_{density}$ = 0 to isolate the ECM fiber realignment effects only.

Anisotropy:

\beq
\frac{da}{dt} = r_{a}(1-a)
\eeq

and

\beq
r_{a} = r_{a0} s_{cell}
\eeq
where r$_{a0}$ is a base rate of change and s$_{cell}$ is the migration speed of the cell changing the voxel anisotropy. 

Fiber Alignment:

\beq
\frac{d\vec f}{dt} = r_R (\vec f - \vec d)
\eeq

and 

\beq
r_R = r_{R0} (1-a)
\eeq

where r$_{R0}$ is a base rate of change and $\vec d$ is the direction of the cell realigning the fiber. The fiber is assumed to have no orientation along itself, but instead cells can travel in either direction. As such, the ECM only influences the original direction of travel

\subsection{ECM Modifying Cell Behavior}\label{sec:ecm_to_cells}

Impact of density on cell:

\beq
s_{migration} = 4 \big((\rho-0.5)^2 + 1.0 \big)S_{max}
\eeq

where s$_{migration}$ is the updated cell migration speed and S$_{max}$ is the maximum speed of a cell. This reflects the concept that with either too little to grab onto or too much to push through, cell speed is impacted by the amount of fiber present.


Impact of anisotropy on cell:

\beq
b_m = a;
\eeq

where b$_m$ is the bias of the motility vector. This reduces the random motility of the cell as fibers it interacts with become more aligned.

Also, the higher the anisotropy, the more a cells motility is directed in the direction of the fibers. See below for more detail.

Impact of fiber alignment on cell:


\beq
\vec  B = a \vec f + (1-a) \vec d
\eeq

where $\vec B$ is the new cell motility bias vector and $\vec d$ is the direction of the cell realigning the fiber.




\end{document}
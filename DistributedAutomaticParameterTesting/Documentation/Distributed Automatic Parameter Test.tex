\documentclass[12pt]{article}

\usepackage[a4paper, total={7in, 9in}]{geometry}
\usepackage{graphicx}
\graphicspath{ {./images/} }
\usepackage{algorithm}
\usepackage{algorithmic}
\usepackage{hyperref}

\setlength{\footskip}{50pt}

\begin{document}
\title{\vspace{-3cm}Distributed Automatic Parameter Testing}
\author{Ben Duggan}
\date{\today}
\maketitle

\section{Setup}
\subsection{Google Sheet}
The Google Sheet can be have any name you wish.  The first 6 columns are set but the remaining columns can be used to store any parameters you wish.  They will be sent as a dictionary in main so make sure they are unique.  The link to the Google Sheet is \url{https://docs.google.com/spreadsheets/d/17QJFFXto0MbOX5dH9GFP3NevNHiuxj7eZf7Pevcg96U/edit?usp=sharing} but is read only.  The first 6 columns are as following:
\begin{itemize}
	\item id: the trial id and must be unique.
	\item status: the current status of the parameter set.  It can be empty meaning not started, in progress or finished.
	\item start date
	\item end date
	\item performed by: the computer that performed the computation as defined by config.txt
	\item comment: any comments either added by the person entering values or main.py
\end{itemize}
After the first six columns you can add settings.  The `PhysiCell\_settings.xml` file is created from `PhysiCell\_settings\_default.xml` and the XML structure must already be present.  The column names are the path to get to the XML variable and the value is what should be set.  There are problems with Google Sheets formating so it's best to set the columns to plain text.

\subsection{Python dependencies}
The python code uses gspread oauth2client and matlab.  You can install them using `pip install gspread oauth2client flask boxsdk boxsdk[jwt]==2.0.0`.

\subsection{sheet.py}
Using the \href{https://developers.google.com/sheets/api/quickstart/python}{Google Sheets API Python Startup Guide} and  \href{https://www.twilio.com/blog/2017/02/an-easy-way-to-read-and-write-to-a-google-spreadsheet-in-python.html?utm_source=youtube&utm_medium=video&utm_campaign=youtube_python_google_sheets}{Google Spreadsheets and Python} guide written by Greg Baugues.\\

Follow the following steps:\\
\begin{enumerate}
	\item Ensure that the dependencies are installed.\\
	\item Create the credentials (this is only needed to create a new set of parameters).
	\begin{enumerate}
		\item Go to \href{https://console.developers.google.com/}{https://console.developers.google.com/}.
		\item Create a new project.
		\item Click Enable API. Search for and enable the Google Drive API and the Google Sheets API.
		\item Create credentials for service account key.  Create a service account and select a Key type of JSON.  
		\item Name the service account and grant it a Project Role of Editor.
		\item Download the JSON file.
		\item Copy the JSON file into DistParam.
	\end{enumerate}
	\item Go to the spreadsheet you wish to use, copy spreedsheet ID (The ID in  \url{https://docs.google.com/spreadsheets/d/17QJFFXto0MbOX5dH9GFP3NevNHiuxj7eZf7Pevcg96U/edit#gid=0} is \textbf{17QJFFXto0MbOX5dH9GFP3NevNHiuxj7eZf7Pevcg96U}) and set it equal to spreedsheetID in sheet.py
\end{enumerate}


\subsection{config.txt}
This is the configuration file for the main file.  It consists of key value pairs separated by a colon.  There can't be a space between the key or value and the colon.  It consists of the following fields:
\begin{itemize}
	\item userName this is the user name of the system.  It is not required but gets saved in the Google Sheet and allows for runs to be traced back to a particular system.  It's default value is "default".
	\item resetTime is the time in seconds at which a parameter set will be added as available if the status hasn't changed from "in progress" to "finished".  This is to ensure that if the computer shuts down or an error occurs the parameter set will still be run.  The default value is "172800" or 2 days.
	\item numOfRuns is the number of trials you want main to run.  If you set it to -1 it will run until there are no more parameters to run.  The default value is "-1".
	\item lastTest is used by main.py to keep track of if a test didn't finish correctly.  If it did not then it will try this set of parameters the next time.  Its default value is "None".
\end{itemize}

\section{Run Test}
To run you must simply type python main.py

\section{How it works}
When you run sheet.py you enter a while loop that breaks when either there are no more parameters or you have reached the maximum number of parameters to run as defined in config.txt.  At the beginning of the while loop block code checks to see if the DB has errors.  Currently this is just checking to see if the code has been running for more than the time given in config.txt.  Next the code fetches the next set of parameters to use.  


\end{document}


\documentclass[12pt]{article}

\usepackage[a4paper, total={7in, 9in}]{geometry}
\usepackage{graphicx}
\graphicspath{ {./images/} }
\usepackage{algorithm}
\usepackage{algorithmic}
\usepackage{hyperref}

\setlength{\footskip}{50pt}

\begin{document}
\title{\vspace{-3cm}Distributed Automatic Parameter Testing}
\author{Ben Duggan}
\date{\today}
\maketitle

\section{Preface}
This code is designed to provide a common tool chain which gets a parameter set, runs that tool chain and uploads the results.  The system is currently designed to grab the next parameter set from Google Sheets, run the tool chain, upload the results to Box and repeat.  The code can be expanded beyond this or not perform some of these steps, this is just how it was designed to be used.

\section{Setup}
\subsection{Python dependencies}
The python code uses gspread, oauth2client, flask and boxsdk.  You can install them using `pip install gspread oauth2client flask boxsdk boxsdk[jwt]==2.0.0`.


\subsection{config.txt}
This is the configuration file for the main file.  It consists of key value pairs separated by a colon.  All fields are optional but you need several to get any functionality.  Here are the keys and what they are for:
\begin{itemize}
	\item userName this is the user name of the system.  It is not required but gets saved in the Google Sheet and allows for runs to be traced back to a particular system.  It's default value is "default".
	\item resetTime is the time in seconds at which a parameter set will be added as available if the status hasn't changed from "in progress" to "finished".  This is to ensure that if the computer shuts down or an error occurs the parameter set will still be run.  The default value is "172800" or 2 days.
	\item numOfRuns is the number of trials you want main to run.  If you set it to -1 it will run until there are no more parameters to run.  The default value is "-1".
	\item lastTest is used by main.py to keep track of if a test didn't finish correctly.  If it did not then it will try this set of parameters the next time.  Its default value is "None".
	
	lastTest:None
	userName:None
	spreedsheetID:None
	client\_id:None
	client\_secret:None
	boxFolderID:None
	removeZip:None
	resetTime:None
	numOfRuns:None
	computerStrength:None
	accessToken:None
	refressToken:None
\end{itemize}

\subsection{Google Sheet}
\subsubsection{Create credentials}
Using the \href{https://developers.google.com/sheets/api/quickstart/python}{Google Sheets API Python Startup Guide} and  \href{https://www.twilio.com/blog/2017/02/an-easy-way-to-read-and-write-to-a-google-spreadsheet-in-python.html?utm_source=youtube&utm_medium=video&utm_campaign=youtube_python_google_sheets}{Google Spreadsheets and Python} guide written by Greg Baugues.\\

Follow the following steps:\\
\begin{enumerate}
	\item Ensure that the dependencies are installed.\\
	\item Create the credentials (this is only needed to create a new set of parameters).
	\begin{enumerate}
		\item Go to \href{https://console.developers.google.com/}{https://console.developers.google.com/}.
		\item Create a new project.
		\item Click Enable API. Search for and enable the Google Drive API and the Google Sheets API.
		\item Create credentials for service account key.  Create a service account and select a Key type of JSON.  
		\item Name the service account and grant it a Project Role of Editor.
		\item Download the JSON file.
		\item Copy the JSON file into DistParam.
	\end{enumerate}
	\item Go to the spreadsheet you wish to use, copy spreedsheet ID (The ID in  \url{https://docs.google.com/spreadsheets/d/17QJFFXto0MbOX5dH9GFP3NevNHiuxj7eZf7Pevcg96U/edit#gid=0} is \textbf{17QJFFXto0MbOX5dH9GFP3NevNHiuxj7eZf7Pevcg96U}) and set it equal to spreedsheetID in config.txt.
\end{enumerate}

\subsubsection{How to use Sheet}
The Google Sheet can be have any name you wish.  The Google Sheet is used like a database with different sheets (tabs) acting as table, the first row of a sheet being the header and the following rows being the entries.  You are only required to have one column which contains ``id'' and is used to reference specific parameter sets.  There are several reserved column names which can be used and will be referenced or populated automatically.  They are shown below:
\begin{itemize}
	\item id: the trial id and must be unique.
	\item status: the current status of the parameter set.  It can be empty meaning not started, in progress or finished.
	\item startTime
	\item endTime
	\item performedBy: the computer that performed the computation as defined by config.txt
	\item comment: any comments either added by the person entering values or main.py
	\item computerStrength: if the computer strength value for a parameter set is greater than the computer strength setting in the config.txt then the set won't be ran on that computer. 
\end{itemize}

You can add any other columns you wish and can access them in the parameter set returned by dap.param.Param.requestParameters().\\


\subsection{Box}
There are two ways to access the Box API: using Oauth 2.0 or Oauth 2.0 with Java Web Tokens (JWT).  Currently only the former is implemented as you need admin access to implement JWT, which individuals don't have if they are using an institution run Box service.  The Oauth 2.0 process works by having an application pass a client id and secret to the server, letting the user login to Box using their credentials and then Box will give an access and refresh token.  The access token is valid for 60 minutes which the refresh token, which gets a new access and refresh token, is good for 60 days.  This means that once you get an access and refresh token you don't have to do it again for 60 days.\\

\subsubsection{Create credentials}


\subsubsection{How to use Box}
To implement this the code must get the access and refresh token from Box.  Once you setup the API and have your client id and secret added to config.txt and you run ........, the code will start a Flask server.  Open a web browser and go to http://127.0.0.1:5000.  Click the link and you will then be asked to enter your Box username and password.  This step is secure and done using Box's services.  Once you click authorize the code will get the access and refresh token, save them to config.txt (assuming the keys are present), close the Flask server and return to the code.  If you are on a server you will need to run the above code on a machine with a display to get an access and refresh token (displayed in your browser or seen in the config file) and then manually enter it in the config file, unless you have access to a display with a browser.  The code can now use Box!\\

\section{Usage}
To run you must simply type python main.py

\subsection{Command line arguments}


\section{How it works}
When you run sheet.py you enter a while loop that breaks when either there are no more parameters or you have reached the maximum number of parameters to run as defined in config.txt.  At the beginning of the while loop block code checks to see if the DB has errors.  Currently this is just checking to see if the code has been running for more than the time given in config.txt.  Next the code fetches the next set of parameters to use.  

\section{DAP Documentation}
This section will go over each file of the DAP module and what they do.\\
\subsection{\_\_init\_\_}
This file contains all code to setup the module and also contains the code to parse the command line arguments.

\subsection{box}
\subsection{config}
\subsection{csv}
\subsection{param}
\subsection{sheet}
\subsection{tools}

\end{document}

